
\thesisacknowledgement

三年的硕士研究生生活转眼就要接近尾声了,而为了组合数学的一次低级失误而懊悔的日子仿佛才刚刚从身边溜走。在校园的时光总是以人们不愿承认的速度在我们面前呼啸而过,回首这三年,除了学到了很多专业知识,但更重要的是,在这段时间中,我开始认真思考自身与世界的关系,思考社会的运作规律并赞叹人类的智慧,思考文化和艺术给个人带来的巨大冲击并对那些伟大的人报以崇高的敬意,“他们在这世界上面不停的奔跑,一不小心就改变了我们;我们在这世界上面不停的奔跑,一不小心就改变了生活”,这些思考让我明白,我们参与社会的方式除了专业,还有常识和勇气,还有“对人类苦难的同情心”,还有对理性的尊重。在我即将离开校园之际,对那些深刻改变了我的思想家,作家,艺术家和企业家,表达我发至内心的感激之情和敬畏之心。

除了那些素未谋面的人,身边的人和事对我生活中的点点滴滴也有着重要的影响。

首先,我要感谢我的导师陈爱国教授,感谢您在学习上对我无私的教导和帮助,让我在学术的海洋里找到自己的方向,也感谢您在生活中对我无微不至的关心和理解,让我倍感温暖。同时,您平时对科学的严谨态度和勤奋努力的精神,我相信实验室的很多兄弟姐妹都有所感触,每当我们开始埋怨睡眠不够或任务量太大时,想起您凌晨三点还在给我们回复微信消息,也立马振奋起了精神;每当我们为学术上的困难准备做出妥协时,您对学术细节的严格把控常常激励着我们迎难而上。您的优秀品质我会永记于心,在往后道路定加倍努力,不辜负您的教导和期望。

我还要感谢郑旭老师,感谢您在学术理论上对我耐心的指导和平日生活中的疑惑的细心解答;在撰写论文过程中,您的意见于我帮助很大,没有您的指导,我无法如此顺利地完成本篇论文。

同时,我要感谢实验室的赵太银老师,罗光春老师和其他所有老师,感谢你们的平易近人,让内向的我逐渐融入到实验室的大家庭;感谢我的朋友和实验室的同学,你们在生活中给与我不求回报的关心和帮助,是你们让我研究生的三年时光变得如此开心和愉快,和你们的相处过程中,也学到你们乐观的品质和思考问题的方式,感谢我们的遇见,愿我们的友谊地久天长!

在这个特殊的时期,我还要感谢所有为新型冠状病毒奋斗在一线的医务人员,特别感谢从广东和海南来到我的家乡洪湖的医疗支援队,是你们大无畏的精神激励着我们,让我们相信春天总会来临;还要感谢在此期间关心我的陈爱国导师、辅导员和我的朋友,是你们的关心消散了我的焦虑情绪。

最后要感谢我的家人,是你们一直养育我、鼓励我和无条件的支持我,在这二十多载的岁月里,你们包容着我的任性和脾气的同时还给与了我无限的爱,我会继续努力,在今后的工作中认真努力,让我们的生活更加幸福!