
\begin{englishabstract}
Cluster analysis is one of the commonly used algorithms in the field of data mining. The clustering algorithm can divide the data set into several subsets, so that the elements within the subsets have some degree of similarity with each other, and between different subsets the similarity of elements is poor; the input of this algorithm is the entire data set, and with the advent of the era of big data, centralized data storage methods have exposed more and more problems, and distributed data storage methods are widely used; In the era of the mobile Internet, user trajectory data has begun to accumulate rapidly. How to perform distributed clustering calculation of trajectory data in a distributed environment has become an urgent problem.

The distributed trajectory clustering algorithm proposed in this paper conducts research on three aspects: network bandwidth consumption, data privacy, and clustering accuracy. The main contents are as follows:

(1) A distributed trajectory clustering algorithm based on composite sampling--CSD-Clustering algorithm is proposed. The algorithm first analyzes the distribution diversity problem of the current distributed trajectory clustering algorithm; the algorithm uses a combination of polynomial fitting and optimization theory to fit the trajectory model for the characteristics of the trajectory data. The idea of the trajectory model fitting methods can well ensure the accuracy of global clustering, and also protect the privacy of trajectory data to a certain extent. In addition, the algorithm also performs a composite sampling scheme on the trajectory data set consisting of trajectory number sampling and trajectory point sampling. The composite sampling scheme can effectively reduce the consumption of network transmission. Finally, the effectiveness and feasibility of the algorithm are verified by simulation experiments. The experimental results show that the CSD-Clustering algorithm can accurately complete the distributed trajectory clustering calculation task, and it also performs better in terms of privacy and bandwidth consumption.

(2) A distributed trajectory clustering algorithm based on Markov chains, the MCD-Clustering algorithm, is proposed. The algorithm first analyzes the problems existing in many distributed clustering algorithms when processing high-dimensional data. The algorithm proposes a scheme to estimate the trajectory sub-cluster using the Markov chain model for the connections between various dimensions in the high-dimensional trajectory data. This idea can well describe the overall data distribution of trajectory clusters with similar trend. Therefore, in order to obtain trajectory subclusters with trajectories of similar trend, the algorithm performs a local clustering operation before training the Markov chain model. The transfer matrix and other related information are transmitted in the network, and the transfer matrix is stored by the sparse matrix storage method, which greatly relieves the pressure of network transmission. Finally, the validity and feasibility of the algorithm is verified by experiments. The experimental results shows that the algorithm has good performance in terms of data privacy, network bandwidth consumption. It has a certain degree of improvement over the CSD-Clustering algorithm in terms of computing performance and data privacy, but it is slightly worse than the CSD-Clustering algorithm in terms of clustering accuracy.
	
	\englishkeyword{Distributed, clustering, data privacy, trajectory data, network bandwidth consumption}
\end{englishabstract}


