	
\begin{chineseabstract}
聚类分析是数据挖掘领域常用的算法之一,聚类算法能够将数据集分成若干个子集,使得子集内的元素彼此之间有着某种程度的相似性,而不同子集之间的元素则相似性较差;这种算法的输入是全体数据集,而随着大数据时代的到来,集中式数据存储方式已然暴露出越来越多的问题,分布式数据存储方式被广泛采用;在移动互联网时代,用户轨迹数据开始快速积累,如何在分布式环境下对轨迹数据进行分布式聚类计算成为亟待解决的问题。

本文提出的分布式轨迹聚类算法,针对网络带宽消耗、数据隐私性和聚类准确度三方面展开研究,主要研究内容如下:

(1)提出了基于复合采样的的分布式轨迹聚类算法——CSD-Clustering算法。算法首先分析了当前分布式轨迹聚类算法存在的分布多样性问题;算法针对轨迹数据的特征采用了多项式拟合与最优化理论结合的方式对轨迹模型进行了拟合,这种轨迹模型拟合的思路能够很好保证全局聚类的准确度,也一定程度地保护了轨迹数据的隐私;除此之外,算法还对轨迹数据集进行了由轨迹数量抽样和轨迹点抽样组成的复合抽样方案,复合抽样方案能够有效减少网络传输的消耗。最后,通过仿真实验对算法的有效性和可行性进行了验证,实验结果表明,CSD-Clustering算法能够准确完成分布式轨迹聚类计算任务,在隐私性和带宽消耗方面也得到了改善。

(2)提出了基于马尔科夫链的分布式轨迹聚类算法——MCD-Clustering算法。算法首先分析了许多分布式聚类算法在处理高维数据时存在的问题;算法针对高维的轨迹数据中各个维度之间的联系,提出了利用马尔科夫链模型估计轨迹子簇的方案,这种思路对于拥有相似走势的轨迹簇能够很好的描述其整体数据分布,故算法为了得到拥有相似轨迹走势的轨迹子簇,在训练马尔科夫链模型之前进行了局部聚类操作;算法在网络中传输转移矩阵以及其他相关信息,并通过稀疏矩阵存储方式来存储转移矩阵,这种方式极大地缓解了网络传输的压力;最后通过实验对算法的有效性和可行性进行了验证,实验结果表明该算法在计算性能和数据隐私方面相对CSD-Clustering算法都有着一定程度的提升,但在聚类准确度方面略差于CSD-Clustering算法。

\chinesekeyword{分布式,聚类,数据隐私,轨迹数据,网络带宽消耗}
\end{chineseabstract}

