\thesischapterexordium

\section{研究工作的背景与意义}

\subsection{研究背景}

聚类算法情况
随着移动互联网时代的到来,数据规模开始成倍地增长,如何从大规模数据中挖掘出有价值的信息成为众多企业和机构需要思考的问题。
聚类算法作为一种数据挖掘技术已被专业人员广泛地应用,聚类算法能够将大量无标签的数据划分成若干个簇,簇中的元素共同包含着某种隐性的特征,通过聚类算法,我们可以发现数据中隐藏着的内在规律,或能够将异常的噪音数据筛选出来;因为聚类算法输入的数据是无标签的,在机器学习中属于无监督学习中的一种。

随着数据规模的逐步增大,数据的集中式存储和管理由于其存在存储能力限制、计算能力限制以及安全方面的考虑,已经不能满足人们的需求,分布式数据库的使用变得日益广泛,在分布式环境下如何进行数据挖掘操作则称为了新的受到人们广泛讨论的话题。针对地理分布多中心联合分析场景,2010到2011年,惠普创新研究计划资助了G-Hadoop项目\citing{wang2013g},旨在开发跨地理分布的分布式Hadoop框架和MapReduce软件框架,以支持数据密集型应用,其尽量用“移动计算”取代“移动数据”,通过在多个集群使用数据并行处理范式,实现可以同时在多个计算中心运行,而无需复制数据。G-Hadoop需要系统下的所有集群都基于Hadoop构建,没有严格限制原始数据的共享。在2012年Jeff Dean在论文\cite{dean2012large}中,引入data parallelism(数据并行)的概念,其早期概念是用于模型训练中,即把训练数据拆分成多份,用每一份独立的训练局部模型,并将多个局部模型之间做综合同步,得到全局模型,通过多轮迭代,逼近最优的全局模型。此外,也有从数据隐私的角度,提出跨域数据联合使用的问题,最典型的是2016年,Google提出了联邦学习架构,至此,基于跨地理中心的数据联合分析,成为研究的热点。

********这段文字是摘录的************
分布式架构上的数据挖掘算法需要相关信息在网络上的传输,涉及到的数据隐私问题引起了人们的广泛关注。调查显示,不管有没有隐私保护措施,17\%的上网者不愿意将自身信息提供给网站,而56\%的调查者只有在好的隐私保护措施下才愿意提供自身的信息给网站\citing{cranor2000beyond}。另外一个方面,在双方或多方合作进行数据挖掘时,由于某种原因,参与者往往不愿意将原始数据与他人共享而只愿意共享数据挖掘的结果。这种情况在科学研究、医学研究及经济和市场动态研究等方面屡见不鲜\citing{Vaidya2002Privacy}\citing{JIAWEI2008数据挖掘概念与技术}。


由此可见,跨属地多中心数据的联合使用,即是面向处理性能,也是面向数据本身的隐私泄露风险问题,在分布式式计算架构研究中,核心面临着计算结果可信、参数共享过程安全、多方激励等诸多问题亟待解决。由于不同类型的计算,涉及到不同的数据关联难题,聚类分析,是其中最难解决的问题。


\subsection{研究意义}
分布式聚类旨在从大规模的、分布式存储的数据中挖掘出分类模型。
针对分布式聚类算法,目前业内的解决思路是:首先在各个计算子节点上进行局部聚类操作,计算子节点将各自计算的结果进行简答处理后发送给中心节点,中心节点依据各个计算子节点传输的局部聚类结果进行全局聚类操作,然后将全局聚类的结果传输给各个计算子节点,计算子节点依据中心节点传输过来的全局聚类结果则可以进行进一步的操作,比如异常判断。计算子节点传输给中心节点的数据一般由统计数据和一小部分代表向量组成,由于代表向量不可能代替本地所有元素信息,这也使得全局聚类的结果准确性不够高;如果代表向量占本地数据比例越来,理论上全局聚类的结果准确度会更高,但这同时给带宽消耗和隐私保护带来了一定的压力。

如上所述,如何在同时考虑带宽消耗和准确度的情况下有效完成分布式聚类算法是目前的研究难点,而且随着移动互联网的普及,用户对于自身隐私问题也越来越关注,分布式环境下,隐私保护问题已成为研究人员不得不考虑的问题。
针对上述分布式聚类面临的技术挑战,本文开展从以下三个方面开展研究内容:
\begin{enumerate}
\item 为了解决分布式聚类的准确度和带宽消耗之间的矛盾,需要探索一种分布式环境中在保证聚类精度的同时尽可能的降低带宽消耗,在有限网络带宽资源情况下实现高准确度的分布式聚类算法。
\item 由于数据量的大规模增长和用户隐私数据的敏感性,网络中传输原始用户数据成为一种非常危险的做法,对于分布式环境中,各个数据中心位于不同地理位置,如何在不传输原始用户数据情况下完成分布式聚类操作是亟待解决的问题。本文将针对这个问题展开研究,旨在探索一种考虑用户隐私保护,同时具备高准确度的分布式聚类算法。
\end{enumerate}


\section{国内外研究现状}


\subsection{分布式聚类算法研究现状}
在分布式聚类算法研究方面,一种传统的分布式密度聚类方法Basic-DDP\citing{he2011mr},将原始数据均匀划分为w份不相交的子集发送至各个子节点进行处理,同时在每一个子节点缓存一份完整的数据集,该方法提高了聚类计算过程中的并行性,但执行聚类算法之前需要将全局数据聚集在一起在实现特定的数据分发,在数据聚集阶段将产生大量的网络流量消耗,而且也缺乏对数据隐私性的考虑。
Januzaj等人基于传统的DBSCAN算法,提出了一种基于密度的分布式聚类算法\citing{Januzaj2004Scalable},该算法主要分为三个步骤:局部聚类、全局聚类和局部聚类模型更新,在局部聚类阶段,各个计算节点首先针对各自本地数据,在设定邻域半径和最小密度的基础上进行DBSCAN聚类算法,DBSCAN聚类算法将输出核心点集合,从核心点集合中选出具有代表性的原型向量,将其传输给中心节点;在全局聚类阶段,中心节点依据各个计算节点传输的原型向量进行全局聚类,将全局聚类结果传输给所有计算节点;最后,各计算节点依据中心节点回传的全局聚类结果更新聚类模型。
文献\cite{kantabutra2000parallel}提出了一种基于k-means的分布式聚类算法,在该算法中K个节点首先计算本地数据的中心向量,然后各节点将自己的中心向量广播给其余的K-1各节点,这样每个节点将受到K各中心向量,各个节点将本地所有数据与K各中心向量进行比较,并将其发送到离该数据最近的中心向量对应的节点。这种算法其准确性与单节点聚类结果一致,但通信开销很大。
文献\cite{郑苗苗2007DK}针对分布式k-means算法通信开销大的问题提出了改进算法——DK-means算法,该算法只需在网络中传输簇心向量和簇的数据个数,大大减少了网络带宽的压力。
文献\cite{李锁花基于特征向量的分布式聚类算法}提出一种新的数据集表达方式——特征向量,主要通过坐标和密度来描述某一密度空间,以较少的数据量反映站点数据的分布特征,并在此基础上提出了一种基于特征向量的分布式聚类算法DCBFV。该算法对于不同的输入参数导致不同的聚类结果,而且当节点数据量很大时,中心节点的计算能力将成为算法的瓶颈。

在隐私保护层面,很多学者针对分布式聚类做出了尝试。
文献\cite{merugu2003privacy}提出了一种考虑隐私保护的分布式聚类算法,在该方算法中,本地节点通过估计概率密度模型来模拟本地数据的生成模型,然后仅在网络中传输生成模型的参数给中心节点,中心节点可以通过本地生成模型参数人工生成数据,然后利用生成的数据对全局概率密度模型进行估计。
文献\cite{张国荣2007分布式环境下保持隐私的聚类挖掘算法}提出了一种适用于数据垂直切分的聚类算法——k-nearest算法,这种算法基于同态加密和随机扰动技术设计了一个安全协议,算法利用半可信第三方参与下的安全求平均值协议,实现了在分布式数据中进行k-means聚类挖掘时隐私保护的要求。
杨林等人\citing{杨林基于}针对基于欧几里得距离的聚类分析隐私保护问题,提出了一种新的隐私保护方法。该方法将安全多方计算协议运用于水平分布和垂直分布两种数据模型上,使得对该两种数据模型进行聚类分析时既满足了保护隐私的前提,又保证了数据间欧几里得距离不变(即挖掘结果的准确性)。
文献\cite{姚瑶一种基于隐私保护的分布式聚类算法}提出的PPDK-means算法是针对水平划分的分布式数据库的。PPDK-means基于K-means的基本思想,对数据进行分布式聚类,在聚类的过程中引入半可 信第三方,并应用安全多方技术保护本站点真实数据不被传送 到其他站点,以达到隐私保护的目的。
从多属地数据隐私保护的角度,Klusch等人提出一种基于密度的分布式聚类方法KDEC\citing{Klusch2003Distributed},在此方法中,各个计算节点利用非参数核密度估计来对本地数据分布进行刻画,将局部概率密度模型传输给中心节点,中心节点依据局部概率模型通过累加操作得到全局概率密度模型,并将其模型广播给各计算节点,计算节点依据全局概率密度模型来实现聚类操作,该算法在网络中传输的是概率密度模型,对隐私保护起到了一定的作用。
Patel等人基于椭圆曲线密码(ECC)提出了一种用于水平切分的分布式k-means算法\citing{patel2015efficient},这种方法避免了在每个站点进行多次加密操作,因此在计算成本方面方面表现不俗,算法采用环形拓扑进行通信,降低了通信成本。


\subsection{轨迹聚类算法研究现状}
同时,由于轨迹数据的复杂性,以上很多分布式聚类方法不能直接扩展到在分布式轨迹聚类应用中。

针对轨迹聚类,其距离度量对于聚类结果至关重要,除了常用的欧氏距离可以自然地扩展到轨迹距离计算上,文献\citing{zhang2006comparison}提出了PCA+距离度量方式,这种距离度量方式相比于欧式距离有更好的抗噪性,但两条轨迹长度必须一致。文献\citing{chen2011clustering}提出了Hausdorff距离,这种度量适用于长度不相等的序列之间的距离计算,然后对于噪声数据十分敏感。Rick等人提出来了LCSS距离度量方式\citing{rick2000efficient},这种度量方式允许轨迹之间存在一定程度的位置偏移。

在轨迹聚类研究方面,Gaffney等人根据统计分析中的期望最大化原则提出了一种混合回归模型\citing{gaffney1999trajectory}。Chudova等人在此研究基础上提出了一种新的混合模型\citing{chudova2003translation},模型中的参数采用对象的空间和时间偏移综合计算得出,更适应于曲线聚类。Alon等人通过动态时间序列对轨迹数据中含有的不同类簇进行分析,通过对象在相邻两个位置转换的马尔可夫模型来表示轨迹簇\citing{alon2003discovering}。Nanni等人根据轨迹时间属性上的差异提出了一种专注于时间的移动对象轨迹聚类算法\citing{nanni2006time},该算法根据轨迹时间段的不同分别对轨迹数据进行轨迹空间相似度度量。Birant等人针对传统DBSCAN算法的不足,提出了一种适用于时空轨迹数据的改进DBSCAN算法\citing{veloso2011urban}。
Palma等人通过基于密度的轨迹聚类算法发现热点位置\citing{Palma2008A},他通过自定义距离度量利用DBSCAN算法找到轨迹中停歇状态和运动状态。Jeung等人使用基于密度的轨迹聚类算法实现了运动物体群体模式的发现\citing{jeung2008discovery},群体模式是许多运动物体同时运动一段时间的活动模式。Hung等人提出了一种框架,这种框架用于发现代表用户频繁运动行为的轨迹路线\citing{ChihClustering}。由于轨迹数据长度一般较长,对于以整个路径为基本单位的聚类算法可能会引发如下两个问题:(1)复杂轨迹中的局部特征可能会被忽略;(2)找不到轨迹的公共子模式。一些研究者针对上述问题提出了基于分段重组的聚类算法。Lee等人提出来了轨迹聚类的分段和重组框架\citing{lee2007trajectory},框架中定义了一种基于最小描述长度(MDL)理论的形式化轨迹划分算法,并提出了子轨迹聚类算法。

Buchin等人\citing{buchin2011segmenting}提出了一系列的时空标准,比如速度、航向、曲率等,在这些标准下提出了一种算法框架,该框架允许用户将轨迹分段成最小数量的片段。以上相关研究,都是面向集中处理场景的。

\subsection{分布式轨迹聚类研究现状}
针对分布式轨迹聚类的研究十分有限,Fan提出了一个更为通用的行为模式定义,并且基于聚类和Apriori算法,实现了分布式系统环境下的SPARE框架\citing{FanA},能够将时空轨迹数据在新定义的行为模式上进行类型划分,但是这个框架是以全局数据的聚集为前提,对全局数据进行特定的数据划分后执行并行计算实现分布式行为模式挖掘没有考虑隐私安全的问题。
文献\cite{肖源分布式车辆时空轨迹异常检测算法研究}提出了一种用于时空轨迹数据流的分布式聚类算法DTC,在文中提出了一种基于多运动特征的轨迹划分方法和包含了位置、中心、形状、
方向、速率和油耗等六个特征量相似性的轨迹结构相似性度量方法。
Wang等人提出一种基于轨迹数据密度分区的分布式并行聚类方法\citing{Wang2018A}。首先将整个轨迹数据集抽象在一个矩形区域内,通过该矩形最长维度的变换将数据合理地划分为若干任务量相当的分区,构建可供分布式并行聚类的局部数据集,然后各工作服务器对局部分区分别执行DBSCAN聚类算法,管理服务器对局部聚类结果进行合并与整合。
Mao等人以减少通信开销提高分布式轨迹流聚类效率为目标提出了一个在线处理分布式轨迹数据流的增量聚类算法(OCluDTS)\citing{Mao2016TSCluWin}。OCluDTS方法使用基于滑动窗口模型的两层分布式框架,通过多个远程节点并行聚类局部轨迹流以及协调者节点合并局部
聚类结果的方式,确保分布式轨迹流聚类获得与集中式方法相同的精度.此外,为了进一步降低oCluDTS算法的总执行开销,提出了仅限于聚类更新的远程节点传输聚类结果给协调者节点以及基于协调者节点相似性计算的剪枝策略等优化措施。
Deng等人提出了一种轨迹大数据聚类算法Tra-POPTICS\citing{deng2015scalable},Tra-POPTICS可以以分布式方式处理轨迹大数据,以满足较好的可扩展性。为了提供一种对轨迹大数据进行聚类的快速解决方案,文章还探索了在图形处理单元(GPGPU)上使用当代通用计算的可行性,使得提出的算法在对轨迹大数据进行聚类时具有很好的可扩展性和计算性能。


\section{研究目标与研究内容}

\section{本论文的结构安排}
本文围绕分布式轨迹聚类的关键技术展开研究,全文结构如下:

第一章,绪论。 首先论述了 本课题的相关背景和研究意义 然后介绍了 当前国内外研究现状 ,最后给出了 本文的主要研究内容与研究目标 以及论文的后续章节安排。

第二章,相关理论和技术。首先介绍了聚类算法,其中详细介绍了k-means算法的算法流程,关于聚类算法的评价指标也进行了介绍。然后就多项式拟合理论基础和最优化理论进行了介绍与分析 最后对马尔可夫链模型进行了相关介绍 为后续研究的进一步展开提供理论依据。

第三章,基于复合采样的的分布式轨迹聚类算法。首先对当前分布式轨迹聚类方案在准确度上的不足进行了分析,随后从聚类准确度、隐私保护和带宽消耗的角度出发,建立了基于复合采样的的分布式轨迹聚类算法。最后对该算法进行了相关实验,并验证了该算法的准确性和有效性。

第四章,基于马尔科夫链的分布式聚类算法。 首先分析了传统分布式轨迹聚类算法存在的缺点,以及第三章中存在的网络负载过重的问题进行了分析,然后马尔科夫链模型建立局部轨迹数据生成模型,对模型建立过程中的观测矩阵、转换概率以及参数训练方法结合最优化理论进行了求解。 最后利用相关数据集进行实验验证证明了该算法的有效性 。

第五章 分布式轨迹聚类系统的设计与实现。首先对分布式计算系统体系架构进行介绍,给出了系统的总体架构设计并分别对系统中的重点模块进行了详细描述 ,展示了系统相应的功能 。

第六 章,总结与展望。 对全文工作进行总结,并探讨未来的研究方向 。
