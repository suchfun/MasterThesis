\chapter{基于马尔科夫链的分布式轨迹聚类算法}

%已研究的缺点:
%1.需要数据聚集,不符合场景和隐私
%2.统计信息+代表向量
%3.概率模型

\section{引言}
针对由局部聚类结果和统计信息组成的数据结构到数据分布的多样性问题,有部分研究采用概率密度分布模型来描述本地数据分布\citing{Klusch2003Distributed}\citing{merugu2003privacy},这类方法可以有效估计低维数据的概率密度分布,但是针对高维数据的概率密度分布则由于训练数据规模大小的限制,在估计高维数据的概率密度分布方面表示很差,这是因为估计数据的概率密度分布需要训练数据尽量多地填充数据所处的空间,而随着数据维度的增大,所需训练数据的数据量需求将会呈指数级增长,现实状况下,训练数据的数据量是很有限的,所以训练高维数据的概率密度分布几乎是不可能完成的任务。而轨迹数据是一种高维数据,故通过概率密度分布模型来估计本地轨迹数据分布是行不通的,直接估计数据的概率密度分布有着默认的先验假设:维度之间是相互独立的。轨迹数据维度之间显然是存在相互关联的,故通过概率密度分布估计轨迹数据的分布是不合适的。若将轨迹数据中的一个坐标点看作是一个维度,那么一条轨迹数据则可以看做是一个高维空间中的一点,本章分析轨迹数据维度之间的相关性,即轨迹点在下一个微小时刻只会向其邻域的点转移,我们抓住这一性质,利用马尔科夫链对轨迹数据进行建模即可以生成和原始数据分布高度相似的轨迹数据集。

对于第三章提出的CSD-Clustering算法,当属地节点上存储轨迹数据规模较大时,即使通过轨迹数量抽样,拟合抽样后的每一条轨迹仍然会产生大量的参数,即CSD-Clustering算法的参数数量会随着轨迹数据量的增多而线性增长(当轨迹数量抽样率和轨迹点数抽样率固定不变时),这对于包含有大规模轨迹数据的属地节点而言意味着较大的网络带宽传输压力,本章提出基于马尔科夫链的分布式轨迹聚类算法(MCD-Clustering算法),依据算法本身的特性,随着轨迹数据的增长,参数数量的增长速度十分有限(当算法参数固定时),这大大缓解了拥有大规模轨迹数据的属地节点在执行聚类操作时的网络传输压力。

在数据隐私保护方面,CSD-Clustering算法因为采用了轨迹数量抽样,未被抽样的轨迹数据隐私保护是十分彻底的,对于抽样轨迹数据,因为采用了轨迹点数抽样和损失函数的正则化策略,生成的轨迹数据也会与原始数据有所差异,但生成轨迹与原始轨迹是存在一一对应关系的,而且一一对应的生成轨迹与原始轨迹在轨迹走势上非常相似,这种算法对轨迹的具体位置有着很好的保护作用,但对于轨迹的整体走势的保护效果则有待提升,而本章提出基于马尔科夫链的分布式轨迹聚类算法(MCD-Clustering算法),只保证轨迹子簇有着相似的分布特征,对于单条轨迹则与原始数据不存在一一对应的关系,这种算法从轨迹子簇层面保护了数据隐私而不是单条轨迹。

针对以上问题提出的基于马尔科夫链的分布式轨迹聚类算法解决了高维轨迹数据分布描述不准确问题,改进了CSD-Clustering算法在网络带宽消耗和隐私保护方面存在的不足,但在聚类准确度上稍逊于CSD-Clustering算法。下面将对MCD-Clustering算法进行详细的阐述,并通过实验验证算法的可行性与有效性。

\section{基于马尔科夫链的分布式聚类算法}

\subsection{问题建模}

首先简述基本的符号定义和距离定义,然后给出聚类准确度、数据隐私性描和设计目标。

假设多属地分布式系统中,有k个跨地理分布的属地节点,包含一个中心节点。每个属地节点拥有m条时空轨迹数据。对于每条时空轨迹,其中有n个数据点,表示数据对象以时间顺序依次经过了n个空间位置。

假设属地节点的轨迹数据集记为D,轨迹数据集包含有n条轨迹,每条轨迹由m个坐标点构成,坐标点的维度为2,即:
\[
\begin{aligned}
D=&\left\{ t_1,t_2,...,t_n \right\} 
\\
t=&\left\{ \left( x_1,y_1 \right) ,\left( x_2,y_2 \right) ,...,\left( x_m,y_m \right) \right\} 
\end{aligned}
\]

轨迹数据通过聚类算法产生的簇集合记为$C=\left\{ C_1,C_2,...,C_k \right\} $,其对应的簇心向量记为$\mu =\left\{ \mu _1,\mu _2,...,\mu _k \right\} $。

若现有轨迹A和轨迹B,且轨迹A与轨迹B长度相等,则这两条轨迹空间距离Dist(A,B)定义为:
\begin{equation}
\label{ch4dist}
Dist(\mathrm{A}, \mathrm{B})=\frac{1}{n} \sum_{i=1}^{n} \sqrt{\left(a_{x}^{i}-b_{x}^{i}\right)^{2}+\left(a_{y}^{i}-b_{y}^{i}\right)^{2}}
\end{equation}

其中$a_x^i$表示轨迹A第i个点在x维度上的数值,$a_y^i$表示轨迹A第i个点在y维度上的数值,$b_x^i$ 、$b_y^i$则分别代表轨迹B第i个点在x、y维度上的数值。	

若轨迹A和轨迹B长度不相等,则两条轨迹距离采用hausdorff距离,其间的距离记为$d_H(A,B)$,表示为:
\begin{equation}
\label{hausdorffdist}
d_{\text{H}}\left( A,B \right) =\max \left\{ \mathop{sup}_{a\in A}\ \mathop{inf}_{b\in B}\ d\left( a,b \right) ,\mathop{sup}_{b\in B}\ \mathop{inf}_{a\in A}\ d\left( a,b \right) \right\} 
\end{equation}

其中a,b分别表示轨迹A和B上的点,d(a,b)表示a点和b点之间的欧氏距离。

本章的设计目的是解决高维轨迹数据分布描述不准确问题,改进CSD-Clustering算法在网络带宽消耗和隐私保护方面存在的不足。为了衡量聚类算法的精度,我们采用了ARI指数,其计算方式如式\ref{ARI}所示,ARI用来评价聚类质量;针对网络带宽的压缩率,我们通过自定义的CR指标来衡量,CR表达式如式\ref{CR}所示。

对于算法隐私保护层面的评价指标,我们从两个方面进行刻画:不确定性和覆盖度。

(1)不确定性:假设攻击者能够从网络传输数据中还原算法模型,从算法模型生成的数据与原始数据的相似程度。

(2)覆盖度:从网络中进行传输的数据中获取的信息量占原始数据信息量的百分比。注意这个度量指标与网络中传输数据量的大小和上述提到的不确定性无关,该指标只与训练模型使用到的数据与原始数据的占比有关。

\subsection{总体方案}

为了避免分布不确定性问题和高维数据分布描述不准确问题,我们提出基于马尔科夫链的分布式聚类算法(MCD-Clustering)。为了避免分布不确定性问题,很自然想到直接传输属地数据概率分布模型参数到中心节点,这种方法在数据量大和数据维度低时使用,而一条具有500个二维坐标点的轨迹数据可以将其看做维度为1000的数据,此时估计数据概率分布模型的方法将不再适用,我们必须通过另外一种模型来估计轨迹数据的概率分布模型。因为每条轨迹各个点之间显然不是独立于彼此的,故当我们将轨迹数据看做是高维数据时,由于每条轨迹数据相邻点存在相关性,可以看做维度和维度之间存在相关性,我们基于这种认知提出使用马尔科夫链来模拟一簇相似轨迹的生成模型,于是提出了基于马尔科夫链的分布式聚类算法(MCD-Clustering)。方案大致流程如图\ref{ch4frame}所示解:
\begin{figure}[H]
	\includegraphics[width=0.8\textwidth]{ch4frame.png}
	\caption{MCD-Clustering 算法框架}
	\label{ch4frame}
\end{figure}

整个方案可以分为四个部分:属地轨迹预处理、属地轨迹生成模型估计、综合求解和属地模型应用。属地轨迹预处理,该阶段分为三个步骤:轨迹子簇生成、轨迹数据网格化和轨迹坐标填充。为了通过马尔科夫链能够更好的估计轨迹生成模型,首先对属地轨迹数据集进行聚类操作来得到多个轨迹子簇,我们假设一个子簇里面的轨迹大部分有着相同的轨迹走势形状。子簇生成后,针对每个子簇中的轨迹进行网格化操作,网格化的作用是通过更少数量的公用坐标点来表示子簇中所有轨迹中的坐标点。由于传感器在采样运动物体轨迹时存在速度的差异,所以经常会出现一条轨迹时间上相邻的坐标点在网格坐标中并不相邻,此时需要在这两个网格中不相邻的点以指定的策略补充一些坐标点,使得所有轨迹中时间上相邻的点在网格中也相邻。属地轨迹生成模型估计,通过马尔科夫链模型来模拟轨迹生成模型,利用属地轨迹数据集得到含有马尔科夫链参数的最大似然函数,利用最优化理论方法求得使最大似然函数取最大值的马尔科夫链参数。综合求解和属地模型应用,属地将求得的马尔科夫链参数传递给中心节点,中心节点通过这些参数重新生成轨迹坐标,利用重新生成的轨迹坐标作为全局聚类的输入,经过全局聚类输出若干簇心向量,将所有的簇心向量传递给各个属地节点,属地节点利用全局聚类计算出的簇心向量来完成各种业务场景应用。


\subsection{属地轨迹预处理}

属地轨迹预处理分为三个阶段:轨迹子簇生成、轨迹数据网格化和轨迹坐标填充。

轨迹子簇生成阶段主要通过聚类算法来实现,本节选择k-means++算法进行聚类,通过聚类算法生成的轨迹子簇记为集合C,其中聚类中使用的距离度量如式\ref{ch4dist}所示。

k-means++聚类在选择k值时需要聚类评价指标来评判不同k值下的聚类效果,这里采用均方误差损失函数作为内部指标,其表达式如式\ref{kmeanscost}所示。
轨迹子簇生成的具体流程如\ref{kmeanspp}所示,这里不再赘述。

轨迹子簇生成后,针对每个子簇进行网格化。原始轨迹数据中的坐标点在二维连续空间上取值,将轨迹数据网格化后,轨迹所有坐标点在二维离散网格空间取值。最简单的网格化则是将二维连续空间中的点坐标取整,即将二维连续空间映射到由所有整数构成的坐标形成的二维离散网格空间,如图\ref{gridization}所示。
\begin{figure}[H]
	\includegraphics[width=0.8\textwidth]{gridization.png}
	\caption{空间网格化}
	\label{gridization}
\end{figure}

对于轨迹数据可以采用同样的方法对其进行网格化,网格化规则可以通过下面式子表示:
$$
\begin{aligned}
f\left( x \right) =\left\{ \begin{array}{c}
	sign\left( x \right) *\left( \lfloor x \rfloor +1 \right) \,\,if\,\,|x-\lfloor x \rfloor |>0.5\\
	sign\left( x \right) *\left( \lfloor x \rfloor -1 \right) \,\,if\,\,|x-\lfloor x \rfloor |\leqslant 0.5\\
\end{array} \right.\\
\varPhi \left( \left( x,y \right) \right) =\left( f\left( x \right) ,f\left( y \right) \right)\\
g\left\{ \left( x_1,y_1 \right) ,\left( x_2,y_2 \right) ,...,\left( x_l,y_l \right) \right\} =\left( x_1,y_1 \right) \,\, \\
if\,\,\left( x_1,y_1 \right) =\left( x_2,y_2 \right) =...=\left( x_l,y_l \right)
\end{aligned}
$$


其中,sign(x)表示实数x的符号。通过上式规则对轨迹中的每一个坐标点进行网格化,图\ref{gridization}展现了轨迹经过网格化后的效果,图\ref{originaltrajectory}表示原始轨迹数据,图\ref{gridedtrajectory}表示网格化后的轨迹数据。网格化对轨迹数据的精度造成了一定程度的影响,但是完全不影响其对轨迹形状的刻画。
\begin{figure}[H]
\subfigure[]{
\label{originaltrajectory}
\includegraphics[width=0.4\textwidth]{originaltrajectory.png}}
\subfigure[]{
\label{gridedtrajectory}
\includegraphics[width=0.4\textwidth]{gridedtrajectory.png}}
\caption{轨迹网格化前后对比}
\label{gridization}
\end{figure}

由于传感器在采样运动物体轨迹时存在不同物体间速度的差异,所以经常会出现一条轨迹时间上相邻的坐标点在网格空间中并不相邻,此时需要在这两个网格中不相邻的点之间通过指定的操作填充一些坐标点,使得所有轨迹中在时间上相邻的点在网格空间中也相邻,这种保证了网格空间数据相邻性的操作称之为轨迹数据填充操作。假设一条网格化后的轨迹数据记为$g=\left\{ \left( x_1,y_1 \right) ,\left( x_2,y_2 \right) ,...,\left( x_l,y_l \right) \right\} $,轨迹数据填充操作流程如下:

\begin{algorithm}[H]
\begin{spacing}{1.2}
 	\KwData{一条网格化的轨迹$g=\left\{ \left( x_1,y_1 \right) ,\left( x_2,y_2 \right) ,...,\left( x_l,y_l \right) \right\} $}
 	\KwResult{轨迹数据填充后的轨迹数据$\hat{g}=\left\{ \left( x_1,y_1 \right) ,\left( x_2,y_2 \right) ,...,\left( x_s,y_s \right) \right\} $}

	 \Repeat{i=1,2,...,l-1}{
	 	\If{$x_i==x_{i+1}$ && $|y_i-y_{i+1}|>1$}{
	 		s = sign($y_{i+1}-y_{i}$)\;
	 		在坐标$(x_i,y_i)$后面填充坐标$(x_i,y_i+s),(x_i,y_i+2*s),...,(x_i,y_{i+1}-s)$\;
	 	}
	 	\ElseIf{$y_i==y_{i+1}$ && $|x_i-x_{i+1}|>1$}{
	 		s = sign($x_{i+1}-x_i$)
	 		在坐标$(x_i,y_i)$后面填充坐标$(x_i+s,y_i),(x_i+2*s,y_i),...,(x_{i+1}-s,y_{i})$\;
	 	} 
	 	\ElseIf{$|x_i-x_{i+1}| \geqslant 1$ && $|y_i-y_{i+1}| \geqslant 1$}{
	 		xs = sign($x_{i+1}-x_i$)\;
	 		ys = sign($y_{i+1}-y_{i}$)\;
	 		在坐标$(x_i,y_i)$后面填充坐标$(x_i+xs,y_i),(x_i+xs,y_i+ys)$\;
	 		\While{假设最后填充的坐标$(x_t,y_t)$不满足:$x_t==x_{i+1}$ || $y_t==y_{i+1}$ || $\left(|x_t-x_{i+1}|==1$ \&\& $|y_t-y_{i+1}|==1\right)$ }{
	 			在坐标$(x_t,y_t)$后面填充坐标$(x_t+xs,y_t),(x_t+xs,y_t+ys)$\;
	 		}
	 		\If{$x_t==x_{i+1}$ && $|y_t-y_{i+1}|>1$}{
	 			s = sign($y_{i+1}-y_{t}$)\;
	 			在坐标$(x_t,y_t)$后面填充坐标$(x_t,y_t+s),(x_t,y_t+2*s),...,(x_t,y_{i+1}-s)$\;
	 		}
	 		\ElseIf{$y_t==y_{i+1}$ && $|x_t-x_{i+1}|>1$}{
	 			s = sign($x_{i+1}-x_t$)\;
	 			在坐标$(x_t,y_t)$后面填充坐标$(x_t+s,y_t),(x_t+2*s,y_t),...,(x_{i+1}-s,y_{t})$\;
	 		}
	 		\ElseIf{$|x_t-x_{i+1}|==1$ && $|y_t-y_{i+1}|==1$}{
	 			s = sign($x_{i+1}-x_t$)\;
	 			在坐标$(x_t,y_t)$后面填充坐标$(x_t+s,y_t)$\;
	 		}
	 	}
	 }
	 \caption{轨迹数据填充操作}
\end{spacing}
\end{algorithm}

\subsection{属地轨迹生成模型估计}
针对轨迹数据网格化和轨迹数据填充后的轨迹子簇,通过马尔科夫链模型来估计其轨迹生成模型。假设子簇C中含有L条轨迹$C=\left\{ t_1,t_2,...,t_L \right\} $,因为每条轨迹都需要经过轨迹数据网格化和轨迹数据填充两个操作,故子簇C中每条轨迹包含的坐标点个数将不再相同。记子簇中轨迹所有坐标点的集合为S,集合S中元素个数为|S|,将集合S中的每一个元素看做是马尔科夫链中的一种状态,其转移矩阵记为P可表示为:
\begin{equation}
\label{transitionmatrix}
P=\begin{array}{c|cccc}
	&		s_1&		s_2&		\dots&		s_{|s|}\\
	\hline
	s_1&		p_{11}&		p_{12}&		\dots&		p_{1|s|}\\
	s_2&		p_{21}&		p_{22}&		\dots&		p_{2|s|}\\
	\vdots&		\vdots&		\vdots&		\ddots&		\vdots\\
	s_{|s|}&		p_{|s|1}&		p_{|s|2}&		\cdots&		p_{|s||s|}\\
\end{array}
\end{equation}

轨迹子簇C对应的似然函数可以表示为:
\[
L=p_{11}^{n_{11}}*p_{12}^{n_{12}}*p_{1|s|}^{n_{1|s|}}*...*p_{|s||s|}^{n_{|s||s|}}
\]

对上式似然函数取对数不影响似然函数取最大值时的参数取值,故轨迹子簇C对应的对数似然函数可以表示为:
\begin{equation}
\label{ch4ll}
LL=n_{11}\ln \left( p_{11} \right) +n_{12}\ln \left( p_{12} \right) +...+n_{|s||s|}\ln \left( p_{|s||s|} \right) 
\end{equation}
求解式\ref{ch4ll}的最大值所得参数矩阵作为马尔科夫链转移矩阵。

为了更清楚上式的求解过程,我们针对其中任意一个状态A的转移概率来进行求解,假设状态A在轨迹子簇C中可以转移到状态B、状态C和状态D,对应的转移概率记为$p_1,p_2和p_3$,且在轨迹子簇C中发生的次数分别记为$n_1,n_2和n_3$,设定轨迹子簇C包含的轨迹数据是充分的,即状态A只能转移到以上三个状态,此时则有:$p_1+p_2+p_3=1$。针对状态A的对数似然函数可以表示为:
\[
LL\left( A \right) =n_1\ln \left( p_1 \right) +n_2\ln \left( p_2 \right) +n3\ln \left( 1-p_1-p_2 \right) 
\]

LL(A)分别对$p_1$和$p_2$求导:
\[
\begin{cases}
	\frac{\partial LL\left( A \right)}{\partial p_1}=\frac{n_1}{p_1}+\frac{-n_3}{1-p_1-p_2}\\
	\begin{array}{c}
	\frac{\partial LL\left( A \right)}{\partial p_2}=\frac{n_2}{p_2}+\frac{-n_3}{1-p_1-p_2}\\
	p_1+p_2+p_3=1\\
\end{array}\\
\end{cases}
\]

通过求解上式方程组可得到方程组的解为:
\[
\begin{cases}
\begin{array}{c}
	\begin{array}{c}
	p_1=\frac{n_1}{n_1+n_2+n_3}\\
	p_2=\frac{n_2}{n_1+n_2+n_3}\\
\end{array}\\
	p_3=\frac{n_3}{n_1+n_2+n_3}\\
\end{array}    
\end{cases}
\]

依照上式解的形式,转移矩阵可以在O(n)时间复杂度内可以得到。

以上构建的模型在轨迹无交叉点的时候能够使用,当轨迹子簇中大部分轨迹形状走势存在交叉点时,我们可以简单分析这种情形下上式模型的不足之处。假设现有轨迹子簇C如图\ref{cross}所示,轨迹依据时间顺序从左下角依次经过A点和B点,最后经过C点到达整个轨迹的右上角,此时轨迹子簇中许多轨迹存在交叉点A,轨迹在A点既可以转移到B点,同样可以转移到C点,而这两种转移将对应两种完全不同的轨迹走势。当轨迹在A点转移到B点时,依据转移矩阵生成轨迹坐标点时会再次经过A点,然后经过C点,此时轨迹形状走势满足整个轨迹子簇形状走势,如图\ref{A2B};而当轨迹在A点转移到B时,依据转移概率矩阵,轨迹将直接经过C点顺势而上到达轨迹右上角,最终将生成许多类似如图\ref{A2C}的轨迹,而这些轨迹与整个轨迹子簇的轨迹形状走势存在较大的差异。
\begin{figure}[H]
\subfigure[]{
\label{cross}
\includegraphics[height=4.5cm,width=4.5cm]{cross.png}}
\subfigure[]{
\label{A2B}
\includegraphics[height=4.5cm,width=4.5cm]{A2B.png}}
\subfigure[]{
\label{A2C}
\includegraphics[height=4.5cm,width=4.5cm]{A2C.png}}
\caption{轨迹子簇中存在交叉点的情形}
\label{crosssituation}
\end{figure}

为了解决轨迹子簇中轨迹存在交叉点情况,我们可以将二维的轨迹想象成三维轨迹在平面上的投影,在二维平面上A点是一个点,但对应到三维空间则是两不同个点,只是它们在平面上的投影相同而已。故我们可以将A点对应的状态分成两个状态,记为A1和A2,若轨迹子簇中存在如图\ref{A2C}的轨迹,则这些轨迹经过A点时状态记为A1,若轨迹子簇中存在如图\ref{A2B}的轨迹,则这些轨迹在经过A点时状态记为A2,将交叉点对应的状态一分为二,后续求解问题则如同轨迹无交叉点的情形一致。通过把交叉点对应状态一分为二的策略,即可在轨迹子簇中轨迹存在交叉点情况下也能准确描述轨迹生成模型。

针对指定的轨迹子簇,其马尔科夫链模型的转移矩阵可以表示为式\ref{transitionmatrix},其状态与轨迹子簇中轨迹坐标点对应,若将马尔科夫链通过图模型表示:所有状态构成图中的点集,状态与状态之间的所有转移构成了图的边集。显然,用于估计轨迹生成模型的马尔科夫链模型对应的图模型不是一个完全图,而是一个稀疏图,因为任意一个状态只能向其在网格空间中相邻的网格点对应的状态进行转移,即任意一个状态只能向极其有限的状态进行转移,这使得马尔科夫链的转移矩阵中存在大量的零元素。于是我们可以通过稀疏矩阵存储方法来代替转移概率矩阵,从而可以减少一部分带宽传输的压力。稀疏矩阵存储方法有很多种,本文采用CSR稀疏矩阵存储方式表示转移矩阵。


% paraDict['lastP'] = lastPoints
% paraDict['firstP'] = firstPoints
% paraDict['transM'] = csr_matrix(transMat)
% paraDict['allS'] = allState
% paraDict['minmaxL'] = [minLength,maxLength]

为了中心节点能够对轨迹数据进行有效的还原,除了传输转移矩阵,还需传输一些额外的信息,包括状态集合、初始状态分布和长度限制。将这些额外信息和转移矩阵打包成的指定的数据包称为基于马尔科夫链的轨迹数据单元(MTD-Unit),其格式如下:
$$
\begin{tabular}{|c|c|c|c|c|}
\hline firstP&lastP&transM&allS&minmaxL\\
\hline
\end{tabular}
$$

数据单元格式字段含义如下:
\begin{itemize}
\item \textbf{allS}:此字段包含了转移矩阵中的所有状态,每一个状态代表了网格空间中的一个坐标点,以二元组的形式表示
\item \textbf{firstP}:此字段表示轨迹初始状态分布情况
\item \textbf{lastP}:此字段表示每条轨迹截至坐标点集合
\item \textbf{minmaxL}:此字段限制了生成轨迹序列时的最大序列长度和最小序列长度
\item \textbf{transM}:此字段表示马尔科夫链对应的转移矩阵的稀疏表示
\end{itemize}

各个属地节点将所有子簇对应的马尔科夫链模型转化成许多MTD-Unit传递给中心节点,中心节点通过这种特殊的数据包能够将轨迹数据进行还原。

\subsection{综合求解和属地模型应用}

各个属地节点将所有轨迹子簇对应的MTD-Unit传输给中心节点,中心节点接收到属地节点传递来的各种参数后,首先将转移概率矩阵稀疏表示还原出原始的轨迹转移矩阵,中心节点通过转移概率矩阵来生成轨迹。
在生成轨迹的过程中,由于多条轨迹之间存在相互交错的情形,故状态转移至原始离散化轨迹对应的截止坐标点不一定意味着生成过程的停止,本算法在生成轨迹停止判断上采用了如下策略:
\begin{itemize}
\item 若当前状态对应的概率分布是无效的(该状态不会转移至任何其他的状态),判断此时生成的轨迹的长度,如果此时长度满足minmaxL字段规定的长度限制,则该轨迹的生成过程停止;如果不满足,则从初始分布开始重新生成轨迹。
\item 若当前状态是截止状态(lastP集合中的状态),判断此时生成的轨迹的长度,如果此时的长度满足minmaxL字段规定的长度限制,则该轨迹的生成过程停止;如果此时生成的轨迹的长度小于minmaxL字段规定的最小长度,则将该状态对应的网格坐标点加入该生成的轨迹中,并继续执行轨迹的生成过程;如果此时生成的轨迹的长度大于minmaxL字段规定的最大长度,则从初始分布开始重新生成轨迹。
\item 若当前状态不是截止状态,则将该状态对应的网格坐标点加入该生成的轨迹中,然后判断此时的生成的轨迹长度是否超过了minmaxL规定的最大长度,若超过了此时生成的轨迹长度超过了最大长度限制,则从初始分布开始重新生成轨迹;若未超过,则继续执行轨迹的生成过程。
\end{itemize}

生成轨迹过程主要流程如下算法\ref{markovGen}所示。\\
\begin{algorithm}[H]
\begin{spacing}{1}
	\KwData{MTD-Unit}
	\KwResult{生成轨迹集合Grs}
	 转移概率矩阵A = MTD-Unit.get('transM')\;
	 状态初始概率分布$\pi$ = MTD-Unit.get('firstP')\;
	 轨迹数量N = $\pi$.getTrNum()\;
	 轨迹截止点集合lastP = MTD-Unit.get('lasatP')\;
	 最小长度minLength = MTD-Unit.get('minmaxL').getMin()\;
	 最大长度maxLength = MTD-Unit.get('minmaxL').getMax()\;
	初始化轨迹序列集合Grs\;
	 \For{i=1,2,,...,N}{
	 	按照状态初始状态分布$\pi$产生状态$now$\;
	 	初始化轨迹序列Gr,并将状态$now$加入Gr\;
	 	\While{True}{
	 		找到状态$now$对应的概率分布$P_{now}$\;
	 		\If{$P_{now}$不符合概率分布的归一约束}{
	 			\If{len(Gr)满足轨迹长度限制}{
	 				break\;
	 			}
	 			\Else{
	 				Gr = []\;
	 				将状态$now$加入Gr\;
	 				continue\;
	 			}
	 		}
	 		依据概率分布$P_{now}$生成下一个状态$next$\;
	 		\If{next in lastP}{
	 			将状态$next$加入Gr\;
	 			\If{len(Gr) < minLength}{
	 				$now$ = $next$\;
	 				continue\;
	 			}
	 			\Elif{len(Gr) > maxLength}{
	 				Gr = []\;
	 				将状态$now$加入Gr\;
	 				continue\; 				
	 			}
	 			\Else{
	 				break\;
	 			}
	 		}
	 		$now$ = $next$\;
	 	}
	 	将Gr加入到Grs\;
	 }
	 \caption{马尔科夫链模型轨迹生成过程}
	 \label{markovGen}
\end{spacing}
\end{algorithm}

依据上述流程即还原出轨迹数据,而此时的轨迹数据长度是不一致的,所以式\ref{ch4dist}定义的距离度量将不再适用,此时我们将采用霍夫曼距离来度量两条轨迹之间的距离,距离度量如式\ref{hausdorffdist}所示。将式\ref{hausdorffdist}定义的距离替换 kmedoids 算法中的欧氏距离,则可以对生成的轨迹数据集进行k-medoids算法聚类操作了。具体聚类流程如算法\ref{kmediodspp}所示。\\
\begin{algorithm}[H]
	 \KwData{样本距离矩阵$M_{nn}$,簇个数K}
	 \KwResult{簇心向量序号$\mu=\left\{\mu_1,\mu_2,...,\mu_K\}\right\}$}
	 从0到n-1中随机选取一个整数$\mu_1$,将其作为第一个簇心对应的元素序号,此时$\mu={\mu_1}$\;
	 D={1,2,...,n-1}\;
	 \Repeat{k=2,3,...,K}{
	 	R=D-$\mu$\;
	 	\For{$t_i$ in R}{
	 		计算$P(t_i)$,$P\left( t_i \right) =\omega \sum_{\mu _i\in \mu}{M_{t_i,\mu_i}^2}$,其中$\omega$是归一化系数\;
	 	}
	 	按照R集合中所有点的概率分布选出$\mu_k$,将其加入集合$\mu$\;
	 }
	 \caption{kmedoidspp初始化}
\end{algorithm}
\begin{algorithm}[H]
	 \KwData{样本距离矩阵$M_{nn}$,簇个数K}
	 \KwResult{簇划分$C=\left\{C_1,C_2,...,C_k\right\}$}
	 $\mu$=kmdoidsspp\_initialization($M_{nn}$,K)\;
	 \Repeat{所有簇心不再更新或更新值小于一定阈值}{
		 将所有的簇置为空,$C_i=\phi \,\,\left( 1\leqslant i\leqslant K \right)$\;
		 \For{j=1,2,...,m}{
		 	依据距离矩阵,计算序号为i的样本与所有簇心之间的距离,并将序号为i的样本归为第$\alpha _i$类,使其满足:
		 	$\alpha _i=arg\ \ \mathop{min}_{\alpha_i \in \mu} M\left[i,\alpha _i\right]$ \;
		 }
		 \For{i=1,2,...,K}{
		 	$\hat{\mu }_i=arg\,\,\mathmathop{min}_{x\in C_i} \left( \sum_{j\in C_i}{M\left[ x,j \right]} \right) $\\
		 	\If{$\hat{\mu}_i\ne \mu _i$}{
		 		$\mu _i$=$\hat{\mu}_i$\;
		 	}
		 }
	 }
	 \caption{k-medoids++ 算法}
	 \label{kmediodspp}
\end{algorithm}

中心节点计算出k个均簇心向量后,将k个簇心向量回传给各个属地节点,当属地节点侦查到新轨迹时,判断其与回传的k个簇心向量的距离来判定是否为异常轨迹:若新轨迹与所有均值向量距离都超过了设定的阈值,则可以判定为轨迹为异常轨迹。


\section{实验与分析}

\subsection{理论分析}

(1)时间复杂度

属地节点时间复杂度分为两个阶段: 轨迹预处理和生成模型参数求解。

假设属地节点含有m条轨迹,每条轨迹含有n个坐标点。轨迹预处理阶段分为轨迹子簇生成个轨迹数据网格化两个步骤。假设轨迹子簇生成中使用的聚类模型为k-means算法,其时间复杂度为O(n*m*k*t),其中k是簇个数,t表示平均迭代次数。轨迹数据网格化,其时间复杂度为O(n*m)。生成模型参数求解,其时间复杂度为O(n*m)。

中心节点时间复杂度分为两个阶段:轨迹生成和全局聚类。假设轨迹数目为m,每条轨迹包含n个点,轨迹生成时间复杂度为O(m*n);全局聚类,其时间复杂度为O(m*n*k*t)。

(2)带宽消耗

带宽分析可以通过每个轨迹子簇需要传输的带宽层面进行分析,假设一个轨迹子簇含有m条轨迹,每条轨迹含有n个坐标点,则将子簇中的轨迹坐标在网络中传输的数据量为m*n*2。通过MTD-Unit数据包传输所需要在网络中传输的数据量记为Q,则网络通信量压缩比为m*n*2/Q。其中,由MTD-Unit数据包字段含义可得,当子簇中的轨迹数量增加,初始分布、转移矩阵和状态集合其大小均不会发生很大的变化,故随着轨迹数量规模的增大,算法将会带来更有效的网络数据压缩能力。

(3)隐私性

算法隐私性层面主要从两个角度进行考量:不确定性和覆盖率。

不确定性通过生成轨迹数据与原始属地轨迹数据之间的相似度进行度量,由马尔科夫链生成轨迹数据过程可知,生成的轨迹数据簇与对应的原始轨迹子簇整体上有着相似的分布,但针对单个轨迹,生成的轨迹可能与原始轨迹子簇中的任意一条轨迹都不相同,故该算法在不确定层面将有着很好的表现。算法在隐私性的覆盖率方面,由于属地节点是针对本地所有轨迹数据来训练马尔科夫链模型,故算法的覆盖率为1。

\subsection{实验与分析}

实验是在真实的数据集上进行的,这些数据集包含了在大阪的ATC购物中心的游客的轨迹。在我们的实验中,我们应用了时间、位置x、位置y、人物id等字段\footnote[1]{https://irc.atr.jp/crest2010_HRI/ATC_dataset/}。经过预处理后,我们选择了2097条轨迹作为原始数据集,每条轨迹包含500个点。

实验系统由三个属地节点和一个中心节点组成。将该算法与集中式聚类方法进行了比较。集中式聚类算法对整个网格化数据集执行标准的k-medoids算法(baseline), baseline的结果视为最优结果,后续试验中使用的ARI指数则是以 baseline聚类结果作为标签。另一种算法则是本章提出的MCD-Clustering算法。

对于两种聚类方案中,都需要对簇个数进行选择,我们以baseline实验中选取的k视为标准,在baseline实验中选取不同的k值其损失函数值变化情况如图\ref{differentK}。
\begin{figure}[H]
	\includegraphics[width=0.8\textwidth]{ck2.png}
	\caption{不同k值损失函数变化}
	\label{differentK}
\end{figure}

通过上图可以确定全局聚类簇个数为8。

各个属地节点完成模型训练完成后,将MTD-Unit传递给中心节点,中心节点依据算法\ref{markovGen}生成轨迹数据,生成后的轨迹数据与原始轨迹数据如图\ref{ch4GenAndOrigin}所示,其中图\ref{gen4All}表示生成的网格化轨迹数据点,图\ref{originAll}表示原始轨迹数据。
\begin{figure}[H]
\subfigure[]{
\label{gen4All}
\includegraphics[width=0.4\textwidth]{ch4disc.png}}
\subfigure[]{
\label{originAll}
\includegraphics[width=0.4\textwidth]{ch4original.png}}
\caption{轨迹生成效果对比图}
\label{ch4GenAndOrigin}
\end{figure}

% Ks = [3,8,10,15,20]
% RI = [0.844,0.878,0.911,0.931,0.981]
% ARI = [0.659, 0.738,0.805, 0.850,0.957]
% CR = [7.344,5.47,5.12,3.96,3.49]



在MCD-Clustering算法中的局部聚类过程,K值选取将会影响马尔科夫链对轨迹生成模型的估计,对于不同K值选取,聚类结果如表\ref{ch4differentK}所示,可以看出随着K的增加,ARI指数稳步上升,但K值的上升同时以为着转移矩阵的个数增加,故网络带宽将会增加,使得CR值有所降低;同时从表\ref{ch4differentK}也可以看出ARI指数相对RI指数在区分粒度上的优势。
\begin{table}[H]
\caption{不同K值对实验效果的影响}
\begin{tabular}{|c|c|c|c|c|c|}
\hline
\diagbox{评价指标}{K} & 3 & 8 & 10  & 15 & 20 \\ \hline
RI   & 0.844 & 0.878 & 0.911 & 0.931 & 0.981\\ \hline
ARI   & 0.659 & 0.738 & 0.805 & 0.85 & 0.957\\ \hline
CR   & 7.344  & 5.47 & 5.12 & 3.96 & 3.49\\ \hline
\end{tabular}
\label{ch4differentK}
\end{table}

图\ref{ch4karicr}展示了局部聚类不同K值下的ARI指数、RI指数和CR值的变化趋势。(保持全局聚类的K值不变)
\begin{figure}[H]
	\includegraphics[width=0.8\textwidth]{ch4ex1.png}
	\caption{不同K值下ARI指数、RI指数和CR值的变化趋势}
	\label{ch4karicr}
\end{figure}

依据图\ref{ch4karicr}和表\ref{ch4differentK},可以看到,随着K值增大,ARI指数也会随之增大,即聚类效果更好。这是因为当K值很小时,轨迹子簇中的轨迹相似性不高,可能会有多种走势形状的轨迹存在,通过这样的轨迹子簇数据训练出来的生成模型会生成多种走势形状混合的轨迹数据,而走势性形状混合的轨迹数据在原始轨迹数据中可能是不存在的,这样就会给后续全局聚类操作造成影响。但随着K值增大,需要在网络中传输更多的转移矩阵,也因此会使网络信息传输压缩率有所降低,但K值增大的同时,子簇更具有相似性,故而转移矩阵也会相应减小。

局部聚类不同K值下,中心节点生成的轨迹数据如图\ref{genTrForDifferentK}所示,其中,图\ref{genk3}表示局部聚类K值取3时中心节点生成的离散化轨迹,图\ref{genk8}表示局部聚类K值取8时中心节点生成的离散化轨迹,图\ref{genk15}表示局部聚类K值取15时中心节点生成的离散化轨迹,图\ref{genk20}表示局部聚类K值取20时中心节点生成的离散化轨迹。通过与图\ref{ch4GenAndOrigin}对比可看出,随着局部聚类K值增大,生成的离散轨迹集合更加接近于原始轨迹数据。
\begin{figure}[H]
\subfigure[]{
\label{genk3}
\includegraphics[width=0.4\textwidth]{gen_disc_k3.png}}
\subfigure[]{
\label{genk8}
\includegraphics[width=0.4\textwidth]{gen_disc_k8.png}}
\subfigure[]{
\label{genk15}
\includegraphics[width=0.4\textwidth]{gen_disc_k15.png}}
\subfigure[]{
\label{genk20}
\includegraphics[width=0.4\textwidth]{gen_disc_k20.png}}
\caption{局部聚类不同K值下生成轨迹效果对比}
\label{genTrForDifferentK}
\end{figure}

为了网络带宽消耗和聚类准确度的平衡,对于局部聚类的K值选择,可以依据K值与损失函数的变化曲线来决定K的取值,实验结果如图\ref{ch4nodeck}所示,图中分别展示了三个属地节点局部聚类K值选择过程:
\begin{figure}[H]
\subfigure[]{
\includegraphics[width=0.3\textwidth]{ch4node1ck.png}}
\subfigure[]{
\includegraphics[width=0.3\textwidth]{ch4node2ck.png}}
\subfigure[]{
\includegraphics[width=0.3\textwidth]{ch4node3ck.png}}
\caption{各节点局部聚类K值与损失函数曲线}
\label{ch4nodeck}
\end{figure}



%在 MCD-Clustering 算法中对轨迹数据进行了网格化处理,假设网格化粒度记为$\delta$,若$\delta$越大则表示网格化粒度越粗,反之,则网格化粒度越精细。我们对不同网格化粒度情况下 MCD-Clustering 算法的聚类效果进行了实验,实验结果如图\ref{deltRIcr}。
%\begin{figure}[H]
%	\includegraphics[width=0.8\textwidth]{DELTA_RI_CR.png}
%	\caption{不同网格粒度对实验结果的影响}
%	\label{deltRIcr}
%\end{figure}

%从上图可以看出,当网格粒度越来越精细,RI指数也会随之增大,即聚类效果越来越好。但随着网格粒度越来越精细,需要在网络中传输的转移矩阵会越来越大,也因此会使网络信息传输压缩率有所降低,相反,网格粒度越粗,需要在网络中传输的转移矩阵会越来越小,网络信息传输压缩率将会有所增长。






\section{本章小结}

本章提出了一种基于马尔科夫链的分布式聚类算法MCD-Clustering,该算法利用马尔科夫链模型来估计轨迹子簇的数据分布,在实现了分布式轨迹聚类的同时也考虑了数据隐私问题和网络带宽消耗问题,相比于第三章的CSD-Clustering算法,在隐私性方面(不确定性和覆盖率)本章提出的算法增强了不确定性,但CSD-Clustering算法在覆盖率层面要优于本章算法;在带宽消耗层面,当数据规模不大时,CSD-Clustering算法由于进行了轨迹数量抽样,其带宽消耗可能略优于本章算法,当数据规模逐渐增大时,本章算法在带宽消耗方面将会优于CSD-Clustering算法。本章通过实验验证了该算法的有效性。
